% Created 2019-02-19 Tue 12:12
% Intended LaTeX compiler: pdflatex
\documentclass[justified, a4paper, notitlepage, captions=tableheading, nobib]{tufte-handout}
        \usepackage{color}
        \usepackage{amssymb}
        \usepackage{amsmath}
        \usepackage{gensymb}
        \usepackage{nicefrac}
        \usepackage{units}
\usepackage[utf8]{inputenc}
\usepackage[T1]{fontenc}
\usepackage{graphicx}
\usepackage{grffile}
\usepackage{longtable}
\usepackage{wrapfig}
\usepackage{rotating}
\usepackage[normalem]{ulem}
\usepackage{amsmath}
\usepackage{textcomp}
\usepackage{amssymb}
\usepackage{capt-of}
\usepackage{hyperref}
\usepackage[utf8]{inputenc}
\usepackage{fontspec}
\usepackage[EU1]{fontenc}
\usepackage{polyglossia}
\setmainlanguage[babelshorthands=true]{german}
\usepackage{atbegshi}
\usepackage{footnote}
\usepackage{minitoc}
\usepackage{booktabs}
\usepackage{longtable}
\usepackage{lmodern}
\usepackage{graphicx}
\usepackage{hyperref}
\usepackage{url}
\usepackage{fancyvrb}
\usepackage{color}
\usepackage{xcolor}
\usepackage{amsmath}
\usepackage{amssymb}
\usepackage{array}
\usepackage{listings}
\usepackage{rotating}
\usepackage{multicol}
\usepackage{pdflscape}
\usepackage{ctable}
\usepackage{parskip}
\usepackage{anysize}
\usepackage{supertabular}
\usepackage{minted}
\usepackage{gensymb}
\usepackage{nicefrac}
\usepackage{units}
\usepackage{siunitx}
\usepackage{marginfix}
\usepackage{hyphenat}
\usepackage{float}
\usepackage{placeins}
\usepackage{tabu}
\usepackage{tabulary}
\usepackage{tocloft}
\usepackage{titlesec}
\usepackage{upquote}
\usepackage{pdfpages}
\usepackage{tabulary}
\usepackage{minted}
\newgeometry{left=0.12\paperwidth,top=1in,headsep=2\baselineskip,
textwidth=0.7\paperwidth,marginparsep=1ex,marginparwidth=0.1\paperwidth,
textheight=44\baselineskip,headheight=\baselineskip}
\usepackage{ifluatex}
\ifluatex
\newcommand{\textls}[2][5]{%
\begingroup\addfontfeatures{LetterSpace=#1}#2\endgroup
}
\renewcommand{\allcapsspacing}[1]{\textls[15]{#1}}
\renewcommand{\smallcapsspacing}[1]{\textls[10]{#1}}
\renewcommand{\allcaps}[1]{\textls[15]{\MakeTextUppercase{#1}}}
\renewcommand{\smallcaps}[1]{\smallcapsspacing{\scshape\MakeTextLowercase{#1}}}
\renewcommand{\textsc}[1]{\smallcapsspacing{\textsmallcaps{#1}}}
\fi
\definecolor{darkblue}{rgb}{0,0,.5}
\definecolor{darkgreen}{rgb}{0,.5,0}
\definecolor{islamicgreen}{rgb}{0.0, 0.56, 0.0}
\definecolor{darkred}{rgb}{0.5,0,0}
\definecolor{mintedbg}{rgb}{0.95,0.95,0.95}
\definecolor{arsenic}{rgb}{0.23, 0.27, 0.29}
\definecolor{prussianblue}{rgb}{0.0, 0.19, 0.33}
\definecolor{coolblack}{rgb}{0.0, 0.18, 0.39}
\definecolor{cobalt}{rgb}{0.0, 0.28, 0.67}
\definecolor{moonstoneblue}{rgb}{0.45, 0.66, 0.76}
\definecolor{aliceblue}{rgb}{0.94, 0.97, 1.0}
\hypersetup{colorlinks=true, breaklinks=true, linkcolor=coolblack, anchorcolor=blue, citecolor=islamicgreen, filecolor=blue,  menucolor=blue,  urlcolor=violet}
\renewcommand\thefootnote{\textcolor{darkred}{\arabic{footnote}}}
\renewcommand{\theFancyVerbLine}{\sffamily\textcolor[rgb]{0.7,0.7,0.7}{\tiny\arabic{FancyVerbLine}}}
\setcounter{secnumdepth}{2}
\titleformat{\section}{\normalfont\Large\bfseries\color{black}} {\llap{\colorbox{coolblack}{\parbox{1.5cm}{\hfill\color{white}\thesection}}}}{1em}{}[]
\titleformat{\subsection}{\normalfont\large\bfseries\color{black}} {\llap{\colorbox{aliceblue}{\parbox{1.5cm}{\hfill\color{coolblack}\thesubsection}}}}{1em}{}[]
\titleformat{\paragraph}{\normalfont\large\bfseries\color{black}} {}{1em}{}[]
\titleformat{\subparagraph}{\normalfont\large\bfseries\color{black}} {}{1em}{}[]
\makeatletter
% Paragraph indentation and separation for normal text
\renewcommand{\@tufte@reset@par}{%
\setlength{\RaggedRightParindent}{0.0pc}%
\setlength{\JustifyingParindent}{0.0pc}%
\setlength{\parindent}{0pc}%
\setlength{\parskip}{6pt}%
}
\@tufte@reset@par
% Paragraph indentation and separation for marginal text
\renewcommand{\@tufte@margin@par}{%
\setlength{\RaggedRightParindent}{0.0pc}%
\setlength{\JustifyingParindent}{0.0pc}%
\setlength{\parindent}{0.0pc}%
\setlength{\parskip}{3pt}%
}
\makeatother
\usepackage{xeCJK}
\setCJKmainfont{YuKyokasho} % for \rmfamily
\setCJKsansfont{YuGothic} % for \sffamily
\setCJKmonofont{Hiragino Maru Gothic ProN}
\setCJKfamilyfont{cjk-vert}[Script=CJK,RawFeature=vertical]{YuKyokasho} % vertical font assignment
\usepackage{ruby}
\renewcommand{\rubysep}{-0.2ex}
\usepackage{booktabs}
\usepackage[normalem]{ulem}
\usepackage{soul}
\setstcolor{red}
\usepackage{csquotes}
\usepackage{hyphenat}
\usepackage[
citestyle=authoryear, %numeric-comp, %authoryear, %verbose,
bibstyle=authoryear,
autocite=inline,
natbib=true,
backend=biber
]{biblatex}
\addbibresource[datatype=bibtex]{jp.bib}
\newcommand\tu[2]{%
\leavevmode
\vtop{\offinterlineskip
\halign{%
\hfil##\hfil\cr % center
\strut#2\cr
\noalign{\kern-.1ex}
\dynscriptsize\strut#1\cr
}%
}%
}
\newcommand\us[2]{$\underset{\text{\footnotesize #2}}{\text{\huge #1}}$}
\newcommand{\kk}[4]{
%\colorbox{green!20}{
\begin{minipage}{1.2cm}
\centering
\\[.5\baselineskip]
{\scriptsize{#3}・\scriptsize{#4}}
\\[.2\baselineskip]
{\huge{#1}}%
\\[-.2\baselineskip]
{{#2}}\\[1.2\baselineskip]
\end{minipage}
%}
}
\author{Göran Kirchner}
\date{\today}
\title{日本語・Japanisch}
\hypersetup{
 pdfauthor={Göran Kirchner},
 pdftitle={日本語・Japanisch},
 pdfkeywords={},
 pdfsubject={},
 pdfcreator={Emacs 26.1 (Org mode 9.2.1)}, 
 pdflang={Germanb}}
\begin{document}

\maketitle
\tableofcontents

\ifxetex
  \newcommand{\textls}[2][5]{%
    \begingroup\addfontfeatures{LetterSpace=#1}#2\endgroup
  }
  \renewcommand{\allcapsspacing}[1]{\textls[15]{#1}}
  \renewcommand{\smallcapsspacing}[1]{\textls[10]{#1}}
  \renewcommand{\allcaps}[1]{\textls[15]{\MakeTextUppercase{#1}}}
  \renewcommand{\smallcaps}[1]{\smallcapsspacing{\scshape\MakeTextLowercase{#1}}}
  \renewcommand{\textsc}[1]{\smallcapsspacing{\textsmallcaps{#1}}}
\fi

\newpage
\section{Kanji }
\label{sec:org78185a5}

\subsection{Tafel }
\label{sec:orgef0abc9}

\paragraph{N5}
\label{sec:org984638f}

\begin{tabular}{ccccc}
\kk{日}{Sonne}{ニチ}{ひ} & \kk{月}{Mond}{ゲツ}{つき} &  &  & \\
\kk{田}{Feld}{デン}{た} & \kk{雨}{Regen}{ウ}{あめ} &  &  & \\
\end{tabular}

\paragraph{N4}
\label{sec:orgd8a31ef}

\paragraph{N3}
\label{sec:orgeced879}

\paragraph{N2}
\label{sec:orgbbb3f9e}

\paragraph{N1}
\label{sec:orgaf08475}

\subsection{Details }
\label{sec:orgd60f3ad}

\paragraph{N5 }
\label{sec:org67741c6}

\begin{note}
Test.
\end{note}

\subparagraph{日}
\label{sec:orgb0bd075}

\label{tab:org3856982}
\begin{tabular}{rll}
1 & sign & 日\\
1 & imi & sun\\
1 & onyomi & ニチ・ジツ\\
1 & kunyomi & ひ・か\\
1 & strokes & 4\\
1 & mnemo & ..\\
\end{tabular}

\label{tab:org54e45b5}
\begin{tabular}{llll}
日曜日 &  にちようび & Sunday & \ruby{日曜日}{にちようび}に\ruby{友人}{ゆうじん}に\ruby{会}{あl}う。\\
\end{tabular}


\subparagraph{月}
\label{sec:orgd0c7a40}

\label{tab:org863a798}
\begin{tabular}{rll}
2 & sign & 月\\
2 & imi & moon\\
2 & onyomi & ゲツ・ガツ\\
2 & kunyomi & つき\\
2 & strokes & 4\\
2 & mnemo & ..\\
\end{tabular}

\subparagraph{雨}
\label{sec:orgc8596c1}

\label{tab:org8efaa1b}
\begin{tabular}{rll}
3 & sign & 雨\\
3 & imi & rain\\
3 & onyomi & ウ\\
3 & kunyomi & あめ・あま\\
3 & strokes & 8\\
3 & mnemo & Rain falls from a big cloud.\\
\end{tabular}

\subparagraph{田}
\label{sec:orgd3d1895}

\label{tab:org546a367}
\begin{tabular}{rll}
4 & sign & 田\\
4 & imi & field\\
4 & onyomi & デン\\
4 & kunyomi & た\\
4 & strokes & 5 \\
4 & mnemo & The rice paddy is laid out to be grid-shaped.\\
\end{tabular}

\paragraph{N4}
\label{sec:org7adefa7}
\paragraph{N3}
\label{sec:org22537f4}
\paragraph{N2}
\label{sec:orgf5d03d8}
\paragraph{N1}
\label{sec:org35d6e0b}
\newpage
\section{Mnemonics}
\label{sec:org292c42a}

\subsection{Piktogramme}
\label{sec:org8bf4d0a}

\paragraph{Natur}
\label{sec:org0794732}

\paragraph{Mensch}
\label{sec:orge07f693}

\paragraph{Körper}
\label{sec:orga514cbb}

\paragraph{Tiere}
\label{sec:org1fef551}

\paragraph{Haus \& Hof}
\label{sec:orgf3493de}

\paragraph{Waffen \& Werkzeuge}
\label{sec:org11f39fa}

\paragraph{Abstrakta}
\label{sec:org3f17b08}

\newpage
\section{Vokabeln }
\label{sec:org9c94e2f}

\subsection{Assimil}
\label{sec:org06f8da4}

\subsection{Linguaphone}
\label{sec:org846cf62}

\subsection{Grundstudium}
\label{sec:orgeeffff9}

\subsection{Grundkurs}
\label{sec:orgb7819ba}

\newpage
\section{Gramatik }
\label{sec:orgf278b0e}

\subsection{Verbflexion }
\label{sec:orgb6acec4}

\paragraph{Konsonantische Verben}
\label{sec:org7304fa5}

\begin{tabular}{rlll}
1 & 本を読む。 & Ich lese ein Buch. & Präs./Futur, lex. Grundform\\
2 & 本を読みます。 & Ich lese ein Buch. & Präs./Futur, masu-Form\\
3 & 本を読んだ。 & Ich las ein Buch. & Perfekt, informell\\
4 & 本を読みました。 & Ich las ein Buch. & Perfekt, höflich\\
5 & 本を読んでいます。 & Ich lese gerade ein Buch. & Aspekte, te-Form\\
6 & 本を読んでしまう。 &  & "`\\
7 & 本を読んである。 &  & "`\\
8 & 本を読んでおく。 &  & "`\\
9 & 本を読んできた。 &  & "`\\
10 & 本を読んでいく。 &  & "`\\
11 & 本を読んでみる。 &  & "` (Versuch, Probe)\\
12 & 本を読んで下さい。 &  & Bitte\\
13 & 本を読みなさい。 &  & Imperativ\\
14 & 本を読め。 &  & "`\\
15 & 本を読もう。 &  & Absicht, Wille\\
16 & 本を読みましょう。 &  & Vorschlag, Aufforderung\\
17 & 本を読むでしょう。 &  & Vermutung\\
18 & 本を読んだことがある。 &  & Erfahrung\\
19 & 本を読んだほうがいい。 &  & Empfehlung\\
20 & 本を読みたい。 &  & Wunsch\\
21 & 本を読んでほしい。 &  & "`\\
22 & 本を読んだり… &  & Exemplativ, Aufzählung\\
23 & 本を読んでもいい。 &  & Erlaubnis\\
24 & 本を読まなければならねい。 &  & Pflicht, Zwang\\
25 & 本を読んではいけない。 &  & Verbot\\
26 & 本を読まなくてもいい。 & Du brauchst das Buch nicht zu lesen. & Erlassung\\
27 & 本が読める。 & Ich kann Bücher lesen. & Potential\\
28 & 本を読むことにする。 & Ich werde etwas lesen. & Entschluß\\
29 & 本を読むことになった。 & Es ergab sich, daß ich Bücher las. & äußerer Umstand\\
30 & 本を読むそうだ。 & Es heißt, er liest. & Hörensagen\\
31 & 本を読みそうだ。 & Mir scheint, er liest ein Buch. & Anschein\\
32 & 本を読まれた。 & Der hat mein Buch gelesen. & Passiv\\
33 & 本を読ませた。 & Ich ließ ihn das Buch lesen. & Kausativ\\
34 & 本を読ませられた。 & Ich mußte das Buch lesen. & Kausativpassiv\\
35 & 本を読めば。 & Wenn man liest \ldots{} & Konditional\\
36 & 本を読んだら。 & Wenn/falls man liest \ldots{} & "`\\
37 & 本を読むなら。 & Wenn/falls du liest \ldots{} & "`\\
38 & 本を読んであげる。 & Ich lese dir aus dem Buch vor. & Besitzwechsel\\
39 & 本を読んでくれた。 & Er las mir aus dem Buch vor. & "`\\
40 & 本を読んでもらった。 & Er las mir aus dem Buch vor. & "`\\
41 & 本を読んでいただく。 & Er liest mir aus dem Buch vor.  & "`\\
\end{tabular}

\paragraph{Vokalische Verben}
\label{sec:org0e5dfe1}

\paragraph{Flexion der i-Adjektive}
\label{sec:org40488da}

\begin{tabular}{llllll}
 &  & lexik. Grundf. &  & desu/masu-Form & \\
\hline
 &  & aff. & neg. & aff. & neg.\\
\hline
prädik. & Präs. &  &  &  & \\
 & Perf. &  &  &  & \\
adnom. & Präs. &  &  &  & \\
 & Perf. &  &  &  & \\
Halbschluß &  &  &  &  & \\
te &  &  &  &  & \\
tara &  &  &  &  & \\
ba &  &  &  &  & \\
\end{tabular}

\paragraph{Flexion der na-Adjektive}
\label{sec:orgaa885b3}

\paragraph{Verbendungen}
\label{sec:org4d989d5}

\subsection{Partikel }
\label{sec:orgc0735e2}

\paragraph{Gegenstand (ha/ga)}
\label{sec:org434c50b}

\paragraph{Tatort (ni/de/wo)}
\label{sec:org61c50ff}

\paragraph{Begrümdung (kara/node/de)}
\label{sec:org6b3ef78}

\newpage
\section{Texte}
\label{sec:org3064e87}

\begin{center}
\rotatebox{-90}{
\begin{minipage}{0.35\textwidth}

\CJKfamily{cjk-vert} %setting font locally
\textbf{三一 \ruby{坂上是則}{さかのうえのこれのり}}
\\[2\baselineskip]
朝ぼらけ
\\[0.5\baselineskip]
\ruby{有明}{ありあけ}の月と 見るまでに
\\[0.5\baselineskip]
\ruby{吉野}{よしの}の里に ふれる\ruby{白雪}{しらゆき}
\end{minipage}
}
\end{center}

\begin{verbatim}
このテキストは、別のフォントを持つことになります。
This is a verbatim environment with mono-space font
\end{verbatim}

\newpage
\section{Links}
\label{sec:org6ad1c77}

\begin{itemize}
\item \href{https://www-user.tu-chemnitz.de/\~heha/j/buchtext/Index.htm}{Grundkurs der modernen japanischen Sprache}
\item \href{https://omp.ub.rub.de/index.php/RUB/catalog/book/31}{Japanisch-Grundkurs in 2 Teilen mit Lehr- und Lernmaterialien}
\end{itemize}

\newpage

\nocite{*}
\printbibliography
\addcontentsline{toc}{section}{\bibname}
\end{document}